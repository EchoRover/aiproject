\documentclass[a4paper, 12pt]{article}

% Packages
\usepackage[utf8]{inputenc}
\usepackage[margin=1in]{geometry}
\usepackage{hyperref}
\usepackage{enumitem}
\usepackage{tcolorbox}

\hypersetup{
    colorlinks=true,
    linkcolor=blue,
    urlcolor=blue
}

% Title
\title{\textbf{Code Documentation}\\\large Parkinson's Disease UPDRS Prediction from Voice Analysis}
\author{Evan Johan Tobias}
\date{December 2025}

\begin{document}

\maketitle
\thispagestyle{empty}

\section*{Repository Access}

\begin{tcolorbox}[colback=blue!5!white,colframe=blue!75!black]
\textbf{GitHub Repository:} \url{https://github.com/EchoRover/aiproject}
\end{tcolorbox}

The complete source code, datasets, notebooks, and figures for this project are available in the above repository.

\section*{Project Structure}

\subsection*{Root Directory}
\begin{itemize}[leftmargin=*]
    \item \texttt{README.md} -- Project overview and setup instructions
    \item \texttt{requirements.txt} -- Python dependencies
    \item \texttt{.venv/} -- Virtual environment (not tracked in git)
\end{itemize}

\subsection*{Datasets (\texttt{datasets/})}
\begin{itemize}[leftmargin=*]
    \item \texttt{parkinsons\_updrs.data} -- Original UCI dataset (5,875 recordings)
\end{itemize}

\subsection*{Notebooks (\texttt{notebooks/parkinsons/})}
Analysis pipeline executed in sequential order:

\begin{enumerate}
    \item \texttt{01\_parkinsons\_eda.ipynb} -- Exploratory Data Analysis
    \begin{itemize}
        \item Target distributions, feature correlations, multicollinearity checks
        \item Disease progression analysis, demographics
    \end{itemize}
    
    \item \texttt{02\_parkinsons\_preprocessing.ipynb} -- Data Preprocessing
    \begin{itemize}
        \item Feature engineering (7 features created, 3 retained)
        \item StandardScaler normalization
        \item 80/20 train/test split with patient stratification
    \end{itemize}
    
    \item \texttt{03\_parkinsons\_regression.ipynb} -- Model Training \& Evaluation
    \begin{itemize}
        \item 5 models: Linear, Polynomial, Decision Tree, Random Forest, Neural Network
        \item Hyperparameter tuning via GridSearchCV
        \item Performance comparison (R\textsuperscript{2}, RMSE, MAE)
    \end{itemize}
\end{enumerate}

\subsection*{Figures (\texttt{figures/})}
\begin{itemize}[leftmargin=*]
    \item \texttt{target\_distributions.png} -- UPDRS score distributions
    \item \texttt{feature\_correlation.png} -- Voice feature correlations
    \item \texttt{model\_comparison\_complete.png} -- Model performance comparison
    \item \texttt{random\_forest\_complete.png} -- Best model analysis
    \item And 6+ additional visualization figures
\end{itemize}

\subsection*{Archive (\texttt{archive/})}
Previous project iterations and alternative analyses (not part of final submission).

\section*{Reproducing Results}

\subsection*{Setup}
\begin{verbatim}
# Clone repository
git clone https://github.com/EchoRover/aiproject.git
cd aiproject

# Create virtual environment
python -m venv .venv
source .venv/bin/activate  # On Windows: .venv\Scripts\activate

# Install dependencies
pip install -r requirements.txt
\end{verbatim}

\subsection*{Run Analysis}
Execute notebooks in order:
\begin{verbatim}
jupyter notebook notebooks/parkinsons/01_parkinsons_eda.ipynb
jupyter notebook notebooks/parkinsons/02_parkinsons_preprocessing.ipynb
jupyter notebook notebooks/parkinsons/03_parkinsons_regression.ipynb
\end{verbatim}

\subsection*{Key Dependencies}
\begin{itemize}[leftmargin=*]
    \item Python 3.8+
    \item pandas, numpy -- Data manipulation
    \item scikit-learn -- ML models, preprocessing
    \item matplotlib, seaborn -- Visualization
    \item torch -- Neural network implementation
\end{itemize}

\section*{Notes}
\begin{itemize}[leftmargin=*]
    \item Random seed set to 42 (chosen because dataset has 42 patients) for reproducibility
    \item Results match those reported in technical paper
    \item Dataset source: \url{https://archive.ics.uci.edu/dataset/189/parkinsons+telemonitoring}
\end{itemize}

\end{document}
